%%%%%%%%%%%%%%%%%%%%%%%%%%%%%%%%%%%%%%%%%%%%%%%%%%%%%%%%%%%%%%%%%%%%%%
% njuthesis 示例模板 v1.4.2 2024-11-08
% https://github.com/nju-lug/NJUThesis
%
% 贡献者
% Yu XIONG @atxy-blip   Yichen ZHAO @FengChendian
% Song GAO @myandeg     Chang MA @glatavento
% Yilun SUN @HermitSun  Yinfeng LIN @linyinfeng
% Yukai Chou @Muzimuzhi
%
% 许可证
% LaTeX Project Public License(版本 1.3c 或更高)
%%%%%%%%%%%%%%%%%%%%%%%%%%%%%%%%%%%%%%%%%%%%%%%%%%%%%%%%%%%%%%%%%%%%%%

%---------------------------------------------------------------------
% 一些提升使用体验的小技巧:
%   1. 请务必使用 UTF-8 编码编写和保存本文档
%   2. 请务必使用 XeLaTeX 或 LuaLaTeX 引擎进行编译
%   3. 不保证接口稳定,写作前一定要留意版本号
%   4. 以百分号(%)开头的内容为注释,可以随意删除
%---------------------------------------------------------------------

%---------------------------------------------------------------------
% 请先阅读使用手册:
% http://mirrors.ctan.org/macros/unicodetex/latex/njuthesis/njuthesis.pdf
%---------------------------------------------------------------------

\documentclass[
    % 模板选项(注意右端逗号):
    %
    % type = bachelor|master|doctor|postdoc, % 文档类型,默认为本科生
    % degree = academic|professional,        % 学位类型,默认为学术型
    %
    % nl-cover,   % 是否需要国家图书馆封面,默认关闭
    % decl-page,  % 是否需要诚信承诺书或原创性声明,默认关闭
    %
    %   页面模式,详见手册说明
    % draft,                  % 开启草稿模式
    % anonymous,              % 开启盲审模式
    % minimal,                % 开启最小化模式
    %
    %   单双面模式,默认为适合印刷的双面模式
    % oneside,                % 单面模式,无空白页
    % twoside,                % 双面模式,每一章从奇数页开始
    %
    %   字体设置,不填写则自动调用系统预装字体,详见手册
    % fontset = win|mac|macoffice|fandol|none,
  ]{njuthesis}

% 模板选项设置,包括个人信息、外观样式等
% 较为冗长且一般不需要反复修改,我们把它放在单独的文件里
\input{njuthesis-setup.def}

% 自行载入所需宏包
% \usepackage{subcaption} % 嵌套小幅图像,比 subfig 和 subfigure 更新更好
% \usepackage{siunitx} % 标准单位符号
% \usepackage{physics} % 物理百宝箱
% \usepackage[version=4]{mhchem} % 绘制分子式
% \usepackage{listings} % 展示代码
% \usepackage{algorithm,algorithmic} % 展示算法伪代码

% 在导言区随意定制所需命令
% \DeclareMathOperator{\spn}{span}
% \NewDocumentCommand\mathbi{m}{\textbf{\em #1}}

% 开始编写论文
\begin{document}

%---------------------------------------------------------------------
%	封面、摘要、前言和目录
%---------------------------------------------------------------------

% 生成封面页
\maketitle

% 模板默认使用 \flushbottom,即底部平齐
% 效果更好,但可能出现 underfull \vbox 信息
% 以下命令用于抑制这些信息
\raggedbottom

\begin{abstract}
随着工业制造智能化进程的加速,产品质量检测的高效性与准确性需求日益提升。传统的人工检测方法因其效率低、成本高、稳定性不足等问题面临着巨大的挑战。近年来,深度学习技术在视觉缺陷检测领域取得了突破性的进展,但大多数方法依赖于大量标注的缺陷数据,而标注数据不仅获得困难,且成本高、泛化能力相当有限,因此在实际工业场景中往往难以实现。与之相反,无监督学习只需要提供正常样本,通过挖掘其特征就能实现缺陷识别,明显更适用于复杂的工业环境,逐渐成为研究的热点。然而,现有的无监督系统通常存在多尺度缺陷检测能力不足、功能模块割裂、参数配置复杂、实时性不足等问题,难以满足工业场景中快速部署的需求。为此,本研究基于开源算法PatchCore,结合工业检测实际需求,设计并实现了一套集样本管理、模型训练与检测功能于一体的轻量化无监督缺陷监测系统,旨在降低对标注数据的依赖,提升检测的灵活性,同时通过模块化设计与功能优化,提升系统的易用性与适应性。该系统创新点主要体现在动态样本组管理、参数抽象映射设计与轻量化交互架构等方面。系统通过多尺度特征融合与参数自适应机制,支持不同尺寸缺陷的精准识别;引入用户导向的参数映射模块,将复杂模型参数简化为"精度-速度-缺陷大小"等直观选项,降低操作门槛;结合数据增强与工程优化策略,提升模型泛化能力与实时性。实际应用中,系统通过热图可视化与分级报告功能,为缺陷成因分析与工艺优化提供支持,具有一定的工业落地价值。
\end{abstract}

\begin{abstract*}
With the acceleration of industrial manufacturing intelligence, there is an increasing demand for efficiency and accuracy in product quality inspection. Traditional manual inspection methods face significant challenges due to their low efficiency, high cost, and poor stability. In recent years, deep learning technology has made breakthrough progress in the field of visual defect detection, but most methods rely on a large amount of annotated defect data, which is not only difficult to obtain but also costly and limited in generalization ability, making it difficult to implement in actual industrial scenarios. In contrast, unsupervised learning only requires normal samples, and can achieve defect identification by mining their features, which is obviously more suitable for complex industrial environments and has gradually become a research hotspot. However, existing unsupervised systems often suffer from insufficient multi-scale defect detection capabilities, fragmented functional modules, complex parameter configuration, and insufficient real-time performance, making it difficult to meet the needs of rapid deployment in industrial scenarios. For this reason, this research is based on the open source algorithm PatchCore and combines the actual needs of industrial inspection to design and implement a lightweight unsupervised defect monitoring system that integrates sample management, model training and detection functions. It aims to reduce the dependence on annotated data and enhance the flexibility of detection, while improving the usability and adaptability of the system through modular design and functional optimization. The innovative points of the system are mainly reflected in dynamic sample group management, parameter abstract mapping design and lightweight interaction architecture. The system supports accurate identification of defects of different sizes through multi-scale feature fusion and parameter adaptive mechanism; introduces a user-oriented parameter mapping module to simplify complex model parameters into intuitive options such as "accuracy-speed-defect size" to lower the operation threshold; combines data enhancement and engineering optimization strategies to improve model generalization ability and real-time performance. In practical applications, the system provides support for defect cause analysis and process optimization through heat map visualization and hierarchical reporting functions, which has certain industrial landing value.
\end{abstract*}

% 生成目录
\tableofcontents
% 生成图片清单
% \listoffigures
% 生成表格清单
% \listoftables

%---------------------------------------------------------------------
%	正文部分
%---------------------------------------------------------------------
\mainmatter

\chapter{引言}

\section{研究背景及意义}

在人类现代社会生活的各个方面,不论是衣食住行,亦或是晨昏四季,工业制品都无处不在。《中国制造 2025》行动纲领指出,建设制造强国任务艰巨而紧迫,需要加速推进信息化与工业化的深度融合,推进生产过程的智能化\cite{preskill2018}。众所周知,在工业制造智能化进程中,产品质量的把控始终是提升工业生产经济效益的关键环节,而这一环离不开产品的缺陷检测。通过缺陷检测能够有效把控产品质量、检测流水线机器的工作状态以及评估生产制造技术的优良,对提高产品质量和生产效率、降低生产成本有着至关重要的作用。因此,基于视觉的工业缺陷检测不仅有非常重要的研究价值,同时也拥有广阔的应用前景。在传统的工业生产过程,缺陷检测主要依靠人工视觉,不仅具有检测效率低、误检率和漏检率高、人工成本高、实时性差、主观误差高的缺点,还有接触损伤的风险。同时,在缺陷尺寸小于 0.5 mm 且无较大光学形变时,人眼检测不到缺陷信息,不适用于大规模工业生产的要求。

后来,随着计算机图像处理技术的突破,机器视觉有效地解决了缺陷检测中人工的弊端。机器视觉检测技术是一种非接触式的自动检测技术,具有安全可靠、检测精度高、可在复杂的生产环境中长时间运行等优点,是实现工厂生产自动化和智能化的一种有效方法。因而机器视觉逐渐取代人工视觉,成为工业缺陷检测的主力。目前,基于机器视觉的缺陷检测技术已广泛应用于工业产品的质量检测、分类检测和包装检测等,涉及钢板、玻璃、印刷、电子、纺织品、零件、木材、钢轨、瓷砖等多种关系国计民生的行业和产品。

然而,基于规则的传统图像处理算法虽然在特定场景下效果稳定,但对于复杂纹理背景和多样化缺陷类型的适应能力较弱,需频繁调整参数。随着深度学习技术的发展,有监督学习方法通过其在理解和提取产品缺陷特征的优势,在检测精度和检测速度上取得了双重突破,但这类方法面临两大问题:一方面,工业场景中缺陷样本稀缺且多样性强,收集大量带标注的异常样本成本高昂;另一方面,工业生产过程的复杂性导致了缺陷模式和类型变幻莫测,有监督模型对未训练过的新型缺陷类型泛化能力有限,难以适应快速变化的生产工艺。

近年来,基于无监督学习的缺陷检测技术因其能自动学习潜在特征和模式,仅需正常样本即可完成训练的特性,逐渐成为研究热点。经研究发现,现有的无监督检测方法存在三个缺陷:第一,计算复杂度高,如部分基于生成对抗网络(GANs)的方法虽然检测精度较高,但运算开销大,难以满足工业实时检测需求;第二,参数配置专业性强,增加了工程部署门槛,不利于非专业用户使用;第三,检测结果信息量少,对于工业制造中的缺陷检测所能提供的辅助作用低下。此外,大多数现有方法仅提供异常评分而不能直接判断异常,需要人工设置阈值,这在动态变化的生产环境中缺乏灵活性。

本文旨在对基于无监督深度学习的缺陷检测算法模型如何向实际工业应用进行转化,以及相应缺陷检测系统的设计与开发展开研究。结合已有的开源算法模型,设计一套面向工业实际应用场景的无监督缺陷检测系统,致力于解决训练数据稀缺且质量参差不齐、模型检测速度与精度难以平衡、模型参数配置复杂、检测效益低下等问题,提高检测效率与准确性,降低检测成本与学习成本,帮助优化生产工艺,从而提高产品质量,提升生产制造的效益,最终实现促进工业智能制造的发展的目标。

\section{国内外研究现状}

工业缺陷检测技术经历了从人工检测到计算机视觉自动检测的发展历程。随着深度学习技术的发展,缺陷检测也经历了从传统图像处理方法到无监督深度学习方法的演进。本节将对工业缺陷检测方法的三个研究阶段进行展开。

\subsection{基于机器视觉的缺陷检测方法}

在深度学习技术尚未兴起之时,产品表面缺陷检测方法可划分为基于传统图像处理的方法和基于机器学习的方法。

传统图像处理主要依靠边缘检测、形态学运算和特征提取等方法。比如,利用阈值分割、边缘检测和形态学操作等技术来提取缺陷区域。这些方法适用于处理简单场景和规则缺陷,在复杂场景中需要频繁调整参数,表现不佳。

机器学习则主要依靠人工提取特征,如SIFT、HOG等,然后使用支持向量机(SVM)、随机森林(Random Forest)等机器学习算法进行分类。然而,这些方法在处理复杂场景时,特征提取和分类器的性能限制了检测的准确性和鲁棒性。

\subsection{基于深度学习的缺陷检测方法}

随着深度学习技术的发展,卷积神经网络(Convolutional Neural Networks, CNN)成为缺陷检测的主要工具,有监督学习方法开始占据工业缺陷检测领域的主导地位。如Faster R-CNN、YOLO系列算法在缺陷定位与分类中取得显著进展,能够在大量标注数据上实现高检测准确率。然而,这些方法需要大量标注数据,而工业缺陷数据难以获取且成本高昂。

\subsection{基于无监督深度学习的缺陷检测方法}

为解决标注数据不足的问题,研究人员开始采用无监督方法,如基于重构误差或特征学习的方法,学习正常样本的分布来检测异常样本。这些方法仅需正常样本即可完成训练,大幅降低数据获取成本。

\section{研究内容与主要工作}

本研究围绕无监督学习的缺陷检测系统的设计与实现展开,基于已发布的开源无监督检测算法SimpleNet和缺陷检测大语言模型AnomalyGPT,重点解决现有系统在用户可用性及易用性、工程落地效益及可行性等方面的不足。核心研究内容与主要工作分为以下四部分:  

\textbf{无监督检测算法优化与工业适配}  

针对工业场景中缺陷样本稀缺的难题,采用以SimpleNet为核心的无监督学习框架,优化算法以适应复杂工业环境。研究重点包括优化多层级特征提取策略,融合卷积神经网络中不同层级的全局与局部特征,增强对微小缺陷的敏感度;设计动态参数调整机制,根据缺陷尺寸与图像分辨率自适应调节特征采样率及比对阈值,实现精度与速度的平衡;开发伪缺陷生成技术,通过模拟亮度异常、随机噪点注入等方式扩充训练数据,提升模型对未知缺陷类型的泛化能力。  

\textbf{多尺度缺陷检测与轻量化工程实现}  

为满足工业场景中多样化缺陷的检测需求,构建多尺度检测体系。通过跨层特征金字塔网络整合宏观表面异常与微观局部缺陷的检测能力;采用模型剪枝、量化压缩及TensorRT加速技术,降低模型计算复杂度,在保证检测精度的同时将推理速度提升至单图0.2-0.5秒;设计客户端-服务器异步架构,分离用户交互与高负载计算任务,利用多线程优化提升系统吞吐量,适配高频产线实时检测需求。  

\textbf{用户友好交互系统设计与开发}  

针对非专业用户的操作痛点,设计直观易用的交互系统。开发参数映射模块,将专业算法参数(如特征维度、采样率)转换为"精度-速度-缺陷大小"等直观选项,降低配置复杂度;集成可视化功能模块,提供训练进度实时监控、缺陷热图叠加显示及分级报告自动生成功能,增强结果可解释性;基于PySide6框架开发跨平台桌面客户端,支持Windows与Linux系统,实现项目管理、样本导入、模型训练与检测的一站式操作流程。  

\textbf{系统验证与工业场景适配性优化}  

通过多维度实验与实际部署验证系统性能。在公开数据集(MVTec AD)与自建工业数据集(涵盖PCB板、金属表面等场景)中测试检测精度与鲁棒性,对比传统阈值分割、有监督模型及商业系统的性能差异;针对真实工业环境中的光照波动、设备震动等干扰,集成图像去噪模块与稳定性增强算法,优化系统在复杂条件下的适应性;开展产线试点部署,收集实际检测数据并迭代优化模型,确保系统在动态生产环境中的可靠运行。  

\section{论文组织结构}

本文围绕无监督学习的缺陷检测系统的设计与实现展开,全文共分为六章,具体组织结构如下:  

\textbf{第一章 引言}  

阐述工业质检智能化转型的背景与挑战,分析传统检测方法及有监督学习技术的局限性,明确无监督学习在缺陷检测中的研究价值。梳理缺陷检测领域的技术发展脉络,对比分析国内外在传统算法、有监督学习及无监督学习方向的研究,总结现有方法的优势与不足,为本研究的突破方向提供理论依据。最后提出本研究的目标与创新点,并概述论文整体结构。  

\textbf{第二章 基本概念和相关工作}  

介绍无监督异常检测的基本概念,包括特征空间、记忆库构建、特征匹配等。分析SimpleNet算法原理,包括特征提取、记忆库构建、异常检测等核心步骤。  

\textbf{第三章 系统需求分析与总体设计}  

结合工业质检场景的实际需求,从功能性与非功能性角度定义系统设计目标。使用客户端-服务器分离架构,明确各模块的交互逻辑与数据流,为后续实现奠定框架基础。   

\textbf{第四章 系统实现与工程化部署}  

描述系统参数映射模块、可视化模块、模型推理优化及数据管理模块的设计与实现,涵盖客户端交互界面开发、服务器端接口设计。最后集成为基于PySide6的跨平台应用。

\textbf{第五章 结果展示与分析}  

展示使用的开源算法SimpleNet以及AnomalyGPT的检测结果,并简要说明为何使用该算法。然后展示本系统的应用效果,并说明本系统在工业质检应用中的优势。  

\textbf{第六章 总结与展望}  

总结本研究的成果,分析其局限性,并提出未来研究方向,包括多算法适配、有监督-无监督混合学习、多模态数据融合及边缘计算部署等拓展路径。

\chapter{基本概念和相关工作}

\section{基本概念}

\subsection{缺陷与异常的概念}

\textbf{缺陷定义}:缺陷是指产品表面或内部的物理瑕疵,如裂纹、划痕、凹陷、污渍等,这些瑕疵可能影响产品的功能或外观。从视觉角度看,缺陷表现为图像中与正常区域存在视觉差异的区域。

工业缺陷可以在一定程度上视为工业产品外表的异常。

\textbf{异常定义}:异常指图像中不符合预期模式范围的数据,包括颜色、纹理或形状的异常变化。异常是一个更广泛的概念,涵盖任何偏离正常情况的图像特征。

\textbf{关系区分}:在工业缺陷检测中,缺陷通常被视为异常的一种具体表现。根据数据之间是否存在上下文关系,异常可分为点异常、上下文异常和集群异常:
\begin{itemize}
    \item \textbf{点异常}:又称为离群值,描述数值上偏离正常样本的独立数据
    \item \textbf{上下文异常}:描述数值属于正常范围但不符合局部上下文规律的数据点
    \item \textbf{集群异常}:描述一系列相关数据的集合,单独数值正常但整体相关性不符合正常模式
\end{itemize}

\subsection{工业缺陷的分类}

基于缺陷出现的位置与表现形式,工业缺陷可分为两大类:

\textbf{表面缺陷}:主要出现在产品表面的局部位置,通常表现为纹理突变、异状区域、反规律模式或错误图案。例如表面裂纹、色块、织物稀织以及商标文字印刷错误等。根据缺陷区域像素值与周围背景的差异性可进一步细分为:
\begin{itemize}
    \item \textbf{离群值型缺陷}:像素值与正常图像有明显差异,如金属表面的突出划痕
    \item \textbf{集群异常型缺陷}:像素值与周围正常区域属于同一范围,但整体模式异常,如织物的细微密度变化
\end{itemize}

\textbf{结构缺陷}:由产品整体结构错误所致,包括形变、错位、缺损与污染。例如电子元件的位置偏移、零部件缺失或组装不当等。这类缺陷涉及到产品的整体结构关系,检测难度通常更高。

\subsection{无监督异常检测}

无监督异常检测旨在通过分析正常样本的特征分布,识别偏离该分布的异常数据,无需依赖标注的缺陷样本。其核心假设是:正常样本在特征空间中呈现紧凑分布,而异常样本则位于分布边缘或外部。

\textbf{工作原理}:在工业缺陷检测中,这类方法通过构建正常样本的"记忆库",计算测试样本与记忆库的差异度(如特征距离、重构误差)生成异常分数,实现缺陷定位。

\textbf{优势}:无监督方法只需使用正常图像进行训练,无须缺陷样本,在数据收集过程中简化了标注需求。在工业场景中,正常样本远多于异常样本,使得无监督方法特别适合实际应用。此外,这类方法不依赖特定缺陷标注,能适应新的检测任务和环境变化。

\section{SimpleNet}

SimpleNet是一种基于无监督学习的工业缺陷检测算法,其核心思想是将正常图像的特征空间与异常图像的特征空间分开,通过在特征空间合成异常样本来提高检测精度。与传统方法相比,SimpleNet在保持高检测精度的同时,显著降低了计算复杂度,适合工业实时场景应用。

\subsection{算法核心思想}

SimpleNet的核心思想包括以下几个方面:
\begin{itemize}
    \item \textbf{预训练特征提取}:利用在ImageNet上预训练的CNN模型作为特征提取器,捕获图像的多尺度特征
    \item \textbf{特征空间合成}:在特征空间而非图像空间进行异常样本合成,更能模拟真实缺陷的特征分布
    \item \textbf{多尺度特征融合}:结合不同CNN层的特征,同时捕获全局语义信息和局部细节信息
    \item \textbf{特征对比学习}:通过对比学习框架,拉近正常样本之间的特征距离,推远正常样本与合成异常样本的特征距离
\end{itemize}

\subsection{适用场景与局限性}

\textbf{适用场景}:
\begin{itemize}
    \item 缺陷样本稀缺或难以获取的工业场景
    \item 需要实时响应的在线检测系统
    \item 具有一定规则性纹理的产品表面检测
    \item 对检测精度和速度都有较高要求的应用
\end{itemize}

\textbf{局限性}:
\begin{itemize}
    \item 对高度不规则纹理背景的检测效果受限
    \item 对极小尺寸缺陷的检测灵敏度不足
    \item 模型训练需要足够数量的正常样本
    \item 参数配置需要专业知识,不易上手
\end{itemize}

\section{AnomalyGPT}

AnomalyGPT是一种结合了大型视觉语言模型与无监督缺陷检测技术的创新方法,它弥补了传统无监督检测系统在直接判断异常与提供解释方面的不足。

\subsection{大模型原理}

AnomalyGPT基于多模态大语言模型,整合了图像理解与自然语言处理能力,实现了从"看到缺陷"到"理解缺陷"的技术跨越。其核心原理包括:
\begin{itemize}
    \item \textbf{视觉-语言对齐}:通过对齐视觉特征和语言特征空间,实现图像内容与文本描述的双向映射
    \item \textbf{缺陷理解与定位}:结合计算机视觉的目标定位能力和语言模型的推理能力,不仅检测缺陷位置,还能理解缺陷类型和特征
    \item \textbf{上下文感知}:利用注意力机制捕捉图像内部区域关系,识别特定上下文中的异常模式
    \item \textbf{少样本学习}:通过预训练获取的广泛知识,能够从少量示例中快速学习特定领域的缺陷模式
\end{itemize}

\subsection{适用场景与局限性}

\textbf{适用场景}:
\begin{itemize}
    \item 需要缺陷解释与报告生成的高级质检系统
    \item 针对复杂产品的多类型缺陷识别
    \item 用户需要与系统进行自然语言交互的场景
    \item 缺陷检测标准需要灵活调整的应用
\end{itemize}

\textbf{局限性}:
\begin{itemize}
    \item 计算资源需求较高,难以部署于资源受限设备
    \item 对于极小或极不明显的缺陷,解释能力有限
    \item 专业领域术语理解需要特定训练
    \item 实时性不及传统算法,不适合高速生产线
\end{itemize}

\section{工业级缺陷检测系统}

商业系统如HALCON、DeepVision已集成无监督检测模块,支持快速部署与多尺度检测。然而,其局限性体现在:
\begin{itemize}
    \item \textbf{算法黑箱化}:参数配置依赖经验,缺乏透明性与可解释性
    \item \textbf{硬件依赖性强}:需高性能GPU支持,难以适配低成本边缘设备
    \item \textbf{扩展性不足}:定制化功能开发门槛高,难以满足中小企业的个性化需求
\end{itemize}

\section{技术栈}

本系统的开发采用了以下技术栈:

\subsection{开发框架}
\begin{itemize}
    \item \textbf{前端框架}:PySide6/Qt - 用于构建跨平台桌面应用界面
    \item \textbf{后端框架}:FastAPI - 提供高性能RESTful API服务
    \item \textbf{深度学习框架}:PyTorch - 支持模型训练与推理
\end{itemize}

\subsection{数据库}
\begin{itemize}
    \item MySQL - 用于存储样本信息、模型配置和检测结果
\end{itemize}

\subsection{部署工具}
\begin{itemize}
    \item PyInstaller - 将Python应用打包为独立可执行文件
    \item Docker - 容器化部署服务端组件,确保环境一致性
\end{itemize}

\subsection{其它工具库}
\begin{itemize}
    \item OpenCV - 图像处理和计算机视觉算法库
    \item NumPy/SciPy - 科学计算和数据处理
    \item DBSCAN - 用于缺陷聚类分析的密度聚类算法
\end{itemize}

%---------------------------------------------------------------------
%	参考文献
%---------------------------------------------------------------------

% 生成参考文献页
\printbibliography

%---------------------------------------------------------------------
%	致谢
%---------------------------------------------------------------------

\begin{acknowledgement}
  感谢导师邬向前教授在研究过程中提供的悉心指导与支持。感谢实验室的各位老师和同学对本研究工作的帮助和建议。特别感谢南京大学软件学院提供的优良科研环境和条件。同时,也要感谢SimpleNet和AnomalyGPT等开源项目的贡献者,他们的工作为本研究奠定了重要基础。
\end{acknowledgement}

%---------------------------------------------------------------------
%	学术简历
%---------------------------------------------------------------------

% 详见手册中"成果列表"一节
% \njuchapter{学术成果}
% \njupaperlist[攻读博士学位期间发表的学术论文]{preskill2018}

%---------------------------------------------------------------------
%	附录部分
%---------------------------------------------------------------------

% 附录部分使用单独的字母序号
\appendix

% 可以在这里插入补充材料
\chapter{正文中涉及的数据及源代码}
\section{系统主要功能模块示例代码}

下面展示系统中参数映射模块的部分核心代码:

\begin{verbatim}
def map_user_parameters(precision, defect_size, speed):
    """
    将用户友好的参数映射到模型专业参数
    
    参数:
        precision: 精度优先级 (1-5)
        defect_size: 缺陷尺寸 (1-5, 1为最小)
        speed: 速度优先级 (1-5)
        
    返回:
        模型参数字典
    """
    # 基础参数映射表
    params = {
        "layers": 3,  # 默认使用3层特征
        "input_size": 224,  # 默认输入尺寸
        "embed_dim": 1024,  # 默认嵌入维度
        "patch_size": 3,  # 默认补丁大小
    }
    
    # 根据精度调整特征层数和嵌入维度
    if precision > 3:
        params["layers"] = 4
        params["embed_dim"] = 2048
    elif precision < 3:
        params["layers"] = 2
        params["embed_dim"] = 512
    
    # 根据缺陷尺寸调整输入尺寸和补丁大小
    if defect_size < 3:  # 小缺陷
        params["input_size"] = 448
        params["patch_size"] = 1
    elif defect_size > 3:  # 大缺陷
        params["input_size"] = 224
        params["patch_size"] = 5
    
    # 根据速度要求微调参数
    if speed > 3:  # 速度优先
        params["input_size"] = max(224, params["input_size"] // 2)
        params["embed_dim"] = min(params["embed_dim"], 1024)
    
    return params
\end{verbatim}

% 完工
\end{document}
